\documentclass[12pt, a4paper]{article} %titlepage 
\usepackage[utf8]{inputenc}
\usepackage[spanish]{babel}
\usepackage{amsmath}
\usepackage{amsfonts}
\usepackage{amssymb}
\usepackage{graphicx}
\usepackage{xcolor}
\usepackage{pstricks}
\usepackage{marginnote}
\usepackage{float}
\usepackage{multicol}
\usepackage[total={18cm,21cm}, centering, left=2.5cm, right=2.5cm, top=2.5cm, bottom=2.5cm]{geometry}

%\definecolor{azul}{rgb}{0.54, 0.17, 0.89}
\definecolor{azul}{rgb}{0.2, 0.2, 0.6} %en la página de latexcolor

\newcommand{\celda}[1]{
    \begin{minipage}{2.5cm}
    \vspace{2mm}
    #1
    \vspace{2mm-}
    \end{minipage}
}



\renewcommand{\baselinestretch }{1.2}

\title{Formato paper}
\author{Piero Valles}
\date{August 2020}

\begin{document}

    \begin{figure}[H]
    	\centering
        %\centering %(si queremos que esté centrado)
        %\raggdright
        \includegraphics[scale=0.12]{logo_ruiz.png}
        \hfill
        \includegraphics[scale=0.12]{logo_ruiz.png}
        \hfill
        \includegraphics[scale=0.12]{logo_ruiz.png}
    \end{figure}
    
    \vspace{2mm}
    
    \begin{center}
        \textbf{\large COMPARACIÓN DE REGLAMENTOS PARA LA PROTECCIÓN AMBIENTAL DEL SECTOR HIDROCARBURO Y MINERO: ¿SON EFECTIVOS?}\\
        \vspace{5mm}
        Valles M. Piero A.$^1$, De la Rosa C. Lexy H.$^2$, Churampi T. Diego J.$^3$\\
    
        \vspace{5mm}
        
        Jorge M. Alison R.$^4$,\\ Delgado V. Luz N.$^5$ \\
        
        \vspace{5mm}
        
        $^1$\textit{Universidad Antonio Ruiz de Montoya. Faculta de Ciencias Sociales. Economía y Gestión Ambiental}\\
        $^2$\textit{Universidad Antonio Ruiz de Montoya. Faculta de Ciencias Sociales. Economía y Gestión Ambiental}\\
        $^3$\textit{Universidad Antonio Ruiz de Montoya. Faculta de Ciencias Sociales}\\
        $^4$\textit{Universidad Antonio Ruiz de Montoya. Faculta de Ciencias Sociales}\\
        $^5$\textit{Universidad Antonio Ruiz de Montoya. Faculta de Ciencias Sociales}\\
    \end{center}
    
    \begin{center}
        \textcolor{blue}{\rule{15cm}{1mm}}
    \end{center}
    
    \begin{abstract}
        En el presente trabajo se realiza una comparación entre dos Reglamentos: Reglamento para la Protección de las Actividades de Hidrocarburos y el Reglamento para Protección Ambiental de la Actividad Minera. Los criterios de comparación se basan en las Autoridades Competentes para cada sector, el nivel de participación ciudadana, los indicadores ambientales contenidos en los EIA. Así como la modificación según sea el caso correspondiente y las características particulares de cada Reglamento. El propósito de este trabajo es determinar la efectividad de las normas para cada sector de extracción para lo cual se utilizarán dos ejemplos de cada uno. Cada ejemplo resaltará el cumplimiento o no de la norma según corresponda, con el fin de determinar si la efectividad de las normas es absoluta. En este sentido, expondremos que la efectividad de los reglamentos dependerá de la temporalidad y de las circunstancias en la que se encuentra la actividad extractiva.
    %\vspace{3mm}\\
    
    \underline{\bf Palabras clave:} \hspace{1mm} {\em  Palabra1, Palabra2, Palabra3.}
    \end{abstract}
    
    \begin{center}
        \textcolor{azul}{\rule{15cm}{1mm}}
    \end{center}

    \begin{multicols}{2}
        \section{Introducción} \vspace{-3mm}
        En el presente trabajo se realiza una comparación entre dos Reglamentos: Reglamento para la Protección de las Actividades de Hidrocarburos y el Reglamento para Protección Ambiental de la Actividad Minera. Los criterios de comparación se basan en las Autoridades Competentes para cada sector, el nivel de participación ciudadana, los indicadores ambientales contenidos en los EIA. Así como la modificación según sea el caso correspondiente y las características particulares de cada Reglamento. El propósito de este trabajo es determinar la efectividad de las normas para cada sector de extracción para lo cual se utilizarán dos ejemplos de cada uno. Cada ejemplo resaltará el cumplimiento o no de la norma según corresponda, con el fin de determinar si la efectividad de las normas es absoluta. En este sentido, expondremos que la efectividad de los reglamentos dependerá de la temporalidad y de las circunstancias en la que se encuentra la actividad extractiva.
        
        \section*{2. Comparación de los reglamentos del sector hidrocarburos y minero}
        Los puntos relevantes para el análisis comparativo entre el Reglamento para la Protección Ambiental en las Actividades de Hidrocarburos (RPAAH) y el Reglamento de Protección Ambiental para las Actividades Mineras (RPAAM) se describen en los siguientes párrafos. \\ En primer lugar, podemos resaltar que ambos reglamentos comparten las mismas autoridades u organismos en materia de fiscalización ambiental como la OEFA, EFAs y el Organismo Fiscalizador Técnico y de Infraestructuras, Osinergmin. Sin embargo, para el caso minero, existen los Gobiernos Regionales y Locales cuya principal función es revisar y aprobar los Estudios Ambientales (EA) de la pequeña minería y de la minería artesanal, mientras que Senace se encarga de la grande y mediana minería. Estos tienen la obligación de velar por el cumplimiento de sus reglamentos correspondientes y de aplicar sanciones para infracciones específicas de acuerdo a sus competencias. Mientras que para el caso del RPAAH, la norma solo establece que los gobierno es una de las autoridades competentes, y en la modificatoria del RPAAH le da un mayor enfoque a su participación en la Garantías de Seriedad de Cumplimiento de los planes de abandono pues estos deben ser establecidos a favor del Ministerio de Energía y Minas o Los Gobiernos Regionales. Asimismo, el Ministerio de Energía y Minas se encarga de fiscalizar las actividades realizadas por la pequeña minería, la minería artesanal y minería informal. En el caso del sector hidrocarburos es el encargado de elaborar, proponer y aplicar la política del sector.\\En segundo lugar, en cuanto a la participación ciudadana, ambos cuentan con intervención de los GORES (gobiernos regionales), pero en el sector minero existe un mayor énfasis en este aspecto, evidentemente porque es una actividad mucho más visible, particularmente la minería de tajo abierto. Además, por ser de tema controversial y de constantes infracciones e incumplimientos en su proceso, ha generado mayores impactos ambientales y sociales que se pudieron evitar o minimizar. Esto último ha suscitado la necesidad de una mayor participación ciudadana; en particular, de la población aledaña a los centros mineros. Sin embargo, aunque la intervención de los GORES se extiende un poco más en la modificatoria del RPAAH, carece de ser una herramienta fundamental y con mayor importancia en comparación al RPAAH, por lo tanto, es necesaria mayor participación ciudadana con información completa y anticipada donde se tomen en cuenta sus opiniones.\\En tercer lugar, en ambas normativas se toman en cuenta lo establecido en los EIA como el LMP, ECAs, etc. Asimismo, el correcto funcionamiento de las infraestructuras al igual que su mantenimiento deben ser los adecuados en todos sus procesos y etapas extractivas. Además, estos reglamentos toman en cuenta los Estudios Ambientales y los Instrumentos de Gestión Ambiental. Sin embargo, los estudios complementarios varían de acuerdo a cada sector. Como mencionamos antes, la participación ciudadana es relevante en el caso de minería pues existe específicamente un Plan de Gestión Social. Este plan busca mitigar los impactos sociales negativos y potenciar los impactos sociales positivos del proyecto minero. \\En cuarto lugar, el RPAAH ha sufrido una modificatoria en artículos específicos mientras que el RPAAM necesitó de un cambio prácticamente total. La modificatoria del RPAAH se centró principalmente en 3 puntos: Contenido del Plan de Abandono, Garantía de Seriedad de Cumplimiento (GSC) y Disposiciones Complementaria transitorias; este último se divide en dos partes, el Plan Ambiental Detallada y el Plazo para la adecuación en caso de Plan de Abandono. A manera de resumen, se puede observar que el grueso de la modificación se centra en fortalecer el Plan de Abandono como un instrumento de gestión ambiental complementario. Si comparamos el RPAAH y la modificatoria podemos observar que el primer caso solo se establecen cuestiones básicas, centrándose principalmente en una definición. Mientras que el segundo le da mayor importancia a las comunidades y Gobiernos Regionales estableciendo la posibilidad de dar usos alternativos a las instalaciones de acuerdo con el titular y el propietario. Una cuestión interesante es que no se especifican o definen con mayor detalle los usos alternativos a los que se refiere. De igual manera, se amplía a mayor detalle la GSC a través de un procedimiento más riguroso para su aprobación y la exigencia de mayor monitoréo de las actividades por parte del titular en lo que dure la elaboración, aprobación y ejecución de los Planes.
        \end{multicols}
        
         \begin{table}[H]  % siempre en la columna izquierda para que no malogre la 
                            % lectura del texto
                \centering
                \begin{tabular}{cccc}  % cccc = 4 de columnas
                \hline                 
                \celda{Nombre} \vline & \celda{Apellido} & \celda{Edad} & \celda{País} \\  
                \hline 
                \celda{Piero} \vline & \celda{Valles} & \celda{24} & \celda{Perú}\\
                \hline
                \celda{De la Rosa} \vline & \celda{Lexy} & \celda{23} & \celda{Perú}\\
                \hline                
                \end{tabular}
                \caption{Ejemplo de crear tabla con datos personales}
                \label{tab:tabla1}
            \end{table}
        
        \begin{multicols}{2}
        En quinto lugar, una característica que diferencia a ambos reglamentos es que para el caso minero existen diversos reglamentos de acuerdo al tipo extracción minera. Es decir, existe un reglamento particular tanto para minería metalúrgica como para no metalúrgica. En estas, se perciben diferentes enfoques en cuanto a la gestión ambiental, dado que la primera genera mayores repercusiones e impactos que la segunda en términos ambientales. Mientras que para el caso de hidrocarburos sólo existe un reglamento que abarca todo tipo de actividad que le corresponde. 
        
        \section*{3. Ejemplos de aplicación}
            \subsection*{3.1 CASO REPSOL}
            Repsol es una empresa multinacional de energía y petroquímica española fundada en 1987. Inició su gestión en Perú el año 1995 y en la actualidad desarrolla actividades de producción, comercialización y exploración de hidrocarburos llevados a cabo en la selva de la región Cusco. Repsol cuenta con varios centros de exploración y explotación a nivel nacional. En cuyas áreas se les dio los derechos de labores exploratorios, presentando en primera instancia EIA-sd que refiere estrictamente a la Perforación de Pozos exploratorios, los cuales han sido aceptados por las autoridades competentes de supervisión y fiscalización. \\
            Esta empresa manifiesta en sus distintos informes la importancia de los mecanismos de sostenibilidad y de la buena gestión ambiental que está llevando a cabo como la gestión de residuos, mitigación de contaminantes, sostenibilidad del aire, agua y biodiversidad, las cuales se encuentran en sus proyectos establecidos en su Política de Seguridad, Salud y Medio Ambiente, en la que se resalta la incorporación de criterios ambientales en todo el ciclo de las actividades, con el propósito de minimizar el impacto sobre el entorno en el que se desarrollan, contando con certificados según la norma ISO 14001 de 21 aeroplantas, 32 instalaciones de suministro de marina y principales certificaciones ISO 14001 por actividades de cierre en  2008, así mismo dispone de una metodología propia, incluida en el Manual de Auditorías de Medio Ambiente y Seguridad, donde se incluye lo tratado en auditorías internas, junto con la realización de auditorías externas (de exigencia legal) y auditorías de certificación.
            \end{multicols}
            
            \begin{figure}[H]
                \centering
                \includegraphics[scale=0.3]{logo_ruiz.png}
                \caption{ejemplo de imagen}
                \label{fig:figura1}
            \end{figure}
            
            \begin{multicols}{2}
            Sin embargo, el 30 de setiembre del 2016 la OEFA reafirma la resolución en la cual declara la responsabilidad administrativa de parte de la empresa Repsol Exploración Perú, en Kinteroni Lote 57 (ubicado en la cuenca de los ríos Ucayali y Madre de Dios en la amazonia, la Convención región de Cusco), dado que realizó actividades de abandono en área de explosivos (casetas de polvorín), área de tanques australianos, poza de perfil sísmico vertical y área de celdas de compostaje, sin contar con un plan de abandono parcialmente aprobado. Estos hechos fueron observados durante una visita de supervisión por parte de la OEFA a la instalación de Kinteroni en el año 2013. Este acto vulnera el Artículo 9 del Reglamento de Protección Ambiental para las Actividades de Hidrocarburos y la Ley General del ambiente. Dicho procedimiento inició en el año 2015, donde se le dio una sanción administrativa por la conducta infractora con una Tipificación de Infracciones y Escala de Sanciones vinculadas con los IGA y Desarrollo de Actividades en Zonas Prohibidas aprobadas por la Resolución General del Ambiente Directivo Nº 049-2013- competente a la OEFA.\\Respecto a ello Repsol dio sus descargas en contra de dicha resolución, en la que afirma que, los EIA-sd y EIA de desarrollo fueron correctamente aprobados, y en cuanto lo referente el plan de abandono parcial del área sí fue presentado al DGAAE en el 2011, y al momento de dicha supervisión por parte la OEFA la solicitud de plan de abandono aún se encontraba en proceso de trámite. Ya para el 2014 se dio por aprobado el Plan de abandono, pero la empresa resalta que sólo era de algunas áreas, puesto que aún se estaba realizando actividades de exploración y reutilización, como parte del cumplimiento del EIA de Desarrollo, rehabilitando y revegetando. Por lo que concluye que no se estaba dando en ese momento un plan de abandono total, además plantea una medida correctiva alternativa, que propone presentar ante el Organismo de Evaluación y Fiscalización Ambiental copia del cargo de presentación hacia el Ministerio de Energía y Minas del Plan de Abandono de la Locación Kinteroni - Lote 57 y evaluar el estado actual del cumplimiento de este. Ante esto la OEFA resuelve que se considera subsanada la conducta infractora, por lo que no corresponde ordenar la realización de una medida correctiva en el presente extremo y queda por archivado el caso, ya que la empresa cuenta con el Plan de abandono establecido en el EIA-sd antes aceptado.\\Se puede visibilizar que la OEFA está cumpliendo su rol de fiscalizar y sancionar, al llevar a cabo actividades de supervisión a las empresas de actividades de hidrocarburo y denunciar en caso exista infracciones establecidas en el Reglamento Ambiental en las Actividades de Hidrocarburos, en este caso la OEFA denuncia el incumplimiento del artículo 9 en el cual se establece: “Los Estudios Ambientales, los Instrumentos de Gestión Ambiental Complementarios, los anexos y demás información complementaria deberán estar suscritos por el Titular y los profesionales responsables de su elaboración”. En este sentido, se denuncia que no se registró un Plan de Abandono Parcial antes de realizar el Plan de Abandono, hecho que es de sanción sin medidas de corrección ya que en casos de actividades que se realicen sin contar con el instrumento de gestión ambiental o la autorización de inicio de operaciones correspondientes, o en zonas prohibidas se pasa a una sanción inmediata. En este caso vemos que ya existía un el Plan de abandono en EIA, pero que también la empresa solicitó este mismo al DGAEE y fue aceptado en el 2014, lo cual es confuso. Además, es de observar el tiempo que conlleva el actuar de la OEFA con una observación desde el 2013, una resolución dictada el 2015 y resuelta el 2016, queda en suspicacias; sin embargo, a comparación de otros casos se podría considerar un tiempo bastante corto. Por último, cabe resaltar que Repsol es una de las pocas empresas en la que hemos encontrado buenas prácticas de sostenibilidad a favor del medio ambiente y que desde el 2017 tiene un Convenio de Cooperación interinstitucional con el MINAM con la finalidad de colaborar e implementar actividades vinculadas a la promoción de la conservación y la recuperación de la biodiversidad.
            \subsection*{3.2 CASO PLUSPETROL}
            La actividad de extracción petrolera en el lote 1AB en Loreto, inició en 1971 con la empresa Occidental Petroleum Company (OXY) hasta aproximadamente 1999. La falta de normativa específica durante este periodo permitía que simplemente se otorguen las concesiones y se dé inicio a la extracción sin especificaciones precisas sobre las consideraciones que se deben tener en los impactos ambientales y centros poblados aledaños. Recién en 2014 se aprueba el Reglamento para la Protección Ambiental en las Actividades de Hidrocarburos (RPAAH). En 1999 ingresa Pluspetrol Norte recibiendo de manera directa de OXY el lote 1Ab, ahora bajo el nombre Lote 192, a través de un traspaso entre empresas privadas (PAUINAMUT, 2018). Se debe tener en cuenta que a pesar de que el RPAAH establece que en “todo traspase o ceder la actividad a un tercero, el adquirente o cesionario está obligado a ejecutar todas las obligaciones ambientales que fueron aprobadas por la Autoridad Ambiental Competente al transferente o cedente “(2014), para entonces las autoridades competentes no habían evaluado el estado en que se encontraban las instalaciones al momento del traspaso.
            
             \begin{figure}[H]
                \centering
                \includegraphics[scale=0.2]{logo_ruiz.png}
                \caption{ejemplo de imagen}
                \label{fig:figura2}
            \end{figure}
            
            Sin embargo, no es hasta el periodo 2000-2015, tiempo en que Pluspetrol extrae petróleo, que esta empresa identifica más de 2000 zonas impactadas. Asimismo, la OEFA también encuentra 100 zonas de impacto adicional a las de Pluspetrol. En este sentido, la OEFA y el Ministerio de Energía y Minas (MINEM) establecen que Pluspetrol es responsable y debe encargarse de las remediaciones respectivas. El problema radica en que terminadas las actividades extractivas de Pluspetrol en 2015, se retira sin aplicar el Plan de Abandono respectivo conforme a lo establecido en el RPAAH.\\De acuerdo al RPAAH, al terminar la actividad extractiva, el Plan de Abandono debe considerar el uso futuro previsible que se le dará al área o lote, de manera que se corrija o minimice cualquier impacto ambiental negativo garantizando así el reacondicionamiento necesario para su uso potencial (RPAAH 2014 y modificatoria). Evidentemente, este es otro punto importante que no se cumple. En primer lugar, Pluspetrol no reconoce en su Plan de Abandono la totalidad de los impactos; hasta 2018 apenas consideró 49 a pesar que las autoridades competentes le asignaron toda la responsabilidad. En segundo lugar, dado que su Plan de Abandono no contempla todos los impactos, el MINEM no aprobó dicho documento lo cual deja alterado y en condiciones paupérrimas el Lote 192. Entre los principales impactos ambientales, solo por mencionar algunos, tenemos: “Vertió aguas altamente salinas de producción en ríos y quebradas, letales para peces y la vida acuática (en 2006 llegó a verter más de 800 mil barriles por día), deforestó grandes áreas de bosque para habilitar canteras e infraestructura (apróx. 2000 ha), etc.” (PAUINAMUT, 2018). Asimismo, la falta de mantenimiento de la infraestructura es otro quebrantamiento directo del RPAAH, cuya sanción aplicada le corresponde a OSINERGMIN. Por otro lado, terminado el contrato de Pluspetrol, la nueva administradora Pacific Exploration y amp; Production expresa que no se hará responsable de los impactos de su antecesora. \\Muchos de las sanciones aplicadas a Pluspetrol han sido derivadas a juicios, los cuales terminan siendo largos y favoreciendo a la empresa. En la actualidad, muchos de los impactos, prácticamente no han sido remediados y los que sí, sólo fueron de forma superficial y no de acuerdo al reglamento. Finalmente, notamos que, aunque el reglamento y su modificación establecen medidas específicas para el mejor cumplimiento de las actividades extractivas de hidrocarburos, los impactos generados en el pasado carecen de una remediación efectiva. No porque exista una debilidad en la norma actual, sino por la facilidad con las que se llevaron a cabo el traspaso de actividades y porque en el pasado no existían normas ambientales apropiadas y de acuerdo al contexto. Evidentemente, lograr que las empresas se adapten rápidamente a una nueva norma, que desde el punto de vista empresarial y de costos, no los favorece, resulta ingenuo.
            \subsection*{3.3 CASO DE MINERA GOLD FIELDS}
            El 16 de diciembre de 2018 en las instalaciones de la empresa minera Gold Fields, productora de oro y cobre, ubicada en la región de Cajamarca, provincia de Hualgayoc, distrito de Hualgayoc, en la comunidad campesina El Tingo se propició una fuga de agua con relave. Lo cual, según reportes del OEFA, el colapso de una de las tuberías de relave de la empresa afectó la quebrada La Hierba, y posteriormente, el río Tingo Maygasbamba. Dado ello, el OEFA y la ANA encontraron presencia de cobre y otros metales en diversos lugares afectados en estas zonas. Y tras varias investigaciones, estas autoridades concluyeron que la causa fundamental del evento consistió en que al alcanzar el embalse del depósito de relaves la cota 3,778 msnm, el relave y el agua del relave con finos de conectaron, mediante una fractura, con el nivel freático, y luego con el sistema de derivación a las demás fuentes de agua. Y posteriormente, el OEFA dictó medidas administrativas en este distrito y provincia de Hualgayoc, en las cuales se ordenan que esta empresa debe implementar en un plazo de 20 días una poza control para el monitoreo del agua provenientes de diferentes manantiales afectados, el cual debe realizarse previo a su entrega a la quebrada La Hierba, con su respectivo cumplimiento de los ECAs. Asimismo, en un plazo de 45 días remediar el suelo, lecho y ribera de la quebrada La Hierba y del río Tingo Maygasbamba, el cual fue afectado por el discurrimiento de aguas de los manantiales afectados sobre este. \\Luego, el 6 de febrero, la población indignada de esta localidad realizó un paro indefinido, bloqueando la carretera, para solicitar que las personas afectadas puedan ser indemnizadas y se realice el tratamiento de las aguas contaminadas. Sin embargo, durante la inspección geológica se pudo constatar que se tomaron medidas inmediatas para mitigar la descarga de agua con contenido de relaves, como: la construcción de 3 pozas de contención y el corte de la tubería y doblado de la misma para elevarla a una cota superior de captación de las aguas remanentes del manantial Las Tomas, y así frenar el flujo. Asimismo, se implementó un canal de conexión hacia la poza de contención de filtraciones Las Águilas para derivar las aguas de la quebrada Las Hierbas hacia ella y su retorno a la presa de relaves. Además, se recogió sedimentos de color plomizo para mermar el grado de contaminación en la cuenca y evitar la propagación del flujo contaminante, incluidos los 36 canales de regadío ubicados en la Microcuenca Tingo Maygasbamba. Finalmente, en esta inspección se observó pocos vestigios del incidente ambiental en superficie, lo que relativamente concuerda con lo que estipuló la empresa. \\En este caso, se puede apreciar que la empresa minera no cumplió con el reglamento ambiental, donde se señala que en sus respectivos depósitos de relaves estas deben implementar medidas de control y monitoreo constante para mantener el balance de agua técnicamente establecido en el depósito de relaves. Sin embargo, esta empresa cumplió con la adopción de recuperación ambiental mediante técnicas que incluyen el manejo de aguas superficiales, control de emisiones, efluentes mineros y el manejo de residuos sólidos del relave. Las cuales se estipulan en su PAMA y en su plan de contingencia frente a casos de contaminación por filtraciones de relaves. Asimismo, es importante en este hecho la presencia de la OEFA, la cual realizó la supervisión correspondiente para determinar las causas de la emergencia ambiental, la responsabilidad de este hecho y de su impacto. En tal sentido, asumió su función, concluyendo que fue un hecho fortuito, por lo cual no se le aplicó a esta empresa una sanción; sin embargo, dictó las ya mencionadas medidas. Del mismo modo, la presencia de la ANA fue factible y bueno con su aporte técnico. \\Finalmente, la Plataforma Nacional de Personas Afectadas por Metales Tóxicos es una forma de expresión de las personas que son vulnerables frente a la existencia de la actividad minera, la cual fue creada con el fin de hacer visible los impactos mineros. Lo cual se observó en el paro que se realizó en esta localidad; sin embargo, la Federación Distrital de Rondas Campesinas manifiesta que dado a que se ha constatado que la empresa ha llegado a un acuerdo con los dueños de las zonas afectadas y se ha iniciado ya la misma reparación ambiental, ya no tiene sentido que personas terceras y ajenas a los afectados estén haciendo este tipo de actividad, buscando chantajear a la minera con el fin de obtener beneficios personales.
            \subsection*{3.4 CASO DE MINERA CERRO VERDE}
            La Sociedad Minera Cerro Verde S.A está localizada en el distrito de Uchumayo, en la provincia de Arequipa, a una altitud promedio de 2600 m.s.n.m, latitud 16.5311 y longitud 71.5992.  En el año 2012 como consecuencia de su incumplimiento de su propio Estudio de Impacto Ambiental, la OEFA impuso una multa de 120 UIT por infringir lo establecido en el artículo 4 de la Resolución Ministerial N. 011-96-EM/VMM, al dejar excesivas cantidades de residuos de metales pesados encontrados en sus efluentes líquidos. Asimismo, quebrantó el artículo 6 del Reglamento de Protección Ambiental para las Actividades Minero Metalúrgicas que hace referencia a los programas de previsión y control contenidos en el EIA y/o Programas de Manejo Ambiental que permitan evaluar y controlar los residuos líquidos y sólidos cuando éstos pudieran tener un efecto negativo sobre el medio ambiente. En ese mismo año fue sancionada por el MINAM a través de la OEFA, debido a la contaminación de los suelos con efluentes de la planta de oxidación y también se le acusa por una deficiente eliminación de polvo en el proceso de chancado. Sin embargo, la empresa realizó una apelación al Tribunal de Fiscalización Ambiental, valiéndose de que no contaban con una planta de oxidación ubicada en la zona donde se hizo la verificación. \\En el año 2015, fue sancionada por la OEFA, por no contar con las medidas de seguridad exigidas para evitar que se expanda de forma desproporcionada el polvo que lanza al medio ambiente en su proceso productivo. Esto contradice el Artículo 12 del Reglamento de Protección Ambiental para las Actividades Minero Metalúrgicas, en el cual se vela por la calidad del aire. Así también, se hace referencia al Artículo 12 sobre la contaminación ambiental por polvos, gases y material particulado. \\En el presente año se denunció a la empresa por el vertimiento de agua contaminada al río Chili, el cual infringe el Artículo 23 del Reglamento de Protección Ambiental para las Actividades de Exploración minera, el cual se refiere al manejo de aguas residuales domésticas e industriales. Como también del Artículo 21 que se refiere al manejo y protección de los cuerpos de agua. Cabe resaltar, que esta problemática viene de años pasados como bien lo señalamos anteriormente, lo cual refleja la falta de responsabilidad y compromiso respecto a la normativa ambiental. Sin embargo, en estos últimos meses, la empresa ha demostrado su interés en la remediación y recuperación de esta fuente de agua. Para ello, ha realizado importantes obras de infraestructura hídrica. Por ejemplo, la instalación de un sistema de captación y tratamiento de aguas residuales.

        \section*{4. ¿Los reglamentos de protección ambiental son efectivos?}
        Determinar la efectividad de cada norma dependerá de algunas condiciones. La primera es la temporalidad del proyecto. Como se evidencia en el caso de PLUSPETROL, los problemas asociados a la extracción, contaminación y determinar la responsabilidad se arrastraron desde antes de establecerse la norma, lo cual generó complicaciones para la correcta fiscalización y cumplimiento de sanciones. Por otro lado, el caso minero muestra que los problemas ambientales se fiscalizaron de acuerdo a la norma y se establecieron las medidas de remediación correspondiente y la evidencia muestran que se está iniciando con los planes de remediación. Sin embargo, puede ocurrir que a pesar de que la intervención fiscalizadora respaldada en la norma sea efectiva, dependerá de la empresa si soluciona el problema o no. \\La ambigüedad en las denuncias por parte del ente fiscalizador puede generar el desperdicio de recursos de las empresas y del Estado. Entonces, se puede percibir que la falta información y capacitación de grupos fiscalizadores particulares puede llevar al incorrecto funcionamiento y cumplimiento de la norma, evitando así su efectividad. \\De acuerdo a la norma muchos de los procedimientos de fiscalización se llevaron a términos, pero la posibilidad de las empresas de apelar a las denuncias extiende los procesos de remediación o recuperación, según sea el caso, de los lugares impactados. Por lo tanto, se debe buscar soluciones más efectivas y específicas que realmente garanticen en su totalidad el cumplimiento de las sanciones. \\El compromiso y la correcta comunicación entre las empresas, población y entes fiscalizadores permite una rápida intervención del problema y la toma de medidas necesarias para la remediación o recuperación ambiental correspondiente. 
        
        \section*{5. Conclusiones}
        En síntesis, después de todo lo visto podemos percibir que existen diferencias en los reglamentos de cada sector, en cuanto a los enfoques que se le da a cada determinante de las actividades extractivas, en circunstancias particulares. Todo ello, teniendo como base el impacto que genera cada actividad y sus influencias con su entorno. \\En cuanto a los casos vistos, nos ha permitido visualizar que en la práctica muchos puntos de la normativa no han sido cumplidos, donde algunas han tomado acciones para reparar los daños ocasionados y otras no. Asimismo, nos ha permitido determinar en qué circunstancias pueden ser efectivas. Se evidencio que la temporalidad resulta relevante cuando se trata de definir a los responsables de los daños ambientales y sus respectivas sanciones. Además, la ambigüedad de las sanciones y la falta de información perjudican la efectividad de las normas. Sin embargo, el compromiso y la comunicación por parte de las empresas con los entes fiscalizadores y entidades ambientales resulta relevante para un mejor accionar frente a los problemas en general que puedan surgir.

        \section*{6. Referencia bibliográficas}
    \end{multicols}
    







\end{document}
